\documentclass{article}
\usepackage{amsmath}
\usepackage[margin=2cm]{geometry}
\usepackage{hyperref}
\usepackage{indentfirst}
\usepackage{setspace}
\doublespacing
\title{Stat222 Research Proposal}
\author{Hui-Fang Ko, Hsin-Wei Tsao}
\date{Feb 3, 2014}
\begin{document}
\maketitle

\section{Introduction}




\section{Resources (Dataset)}
\subsection{Restaurant Scores in San Francisco}
This is a dataset provided by the San Francisco of Department of Public Health at \url{https://data.sfgov.org/Public-Health/Restaurant-Scores/stya-26eb}. The Health Department has developed an inspection report and scoring system. After conducting an inspection of the facility, the Health Inspector calculates a score based on the violations observed. So this dataset contains information of 6073 restaurants in San Francisco area and their score in the inspection conducted by the department.

\subsection{Yelp API}
With Yelp API, we can get the consumers' reviews of those restaurants listed in the dataset above. Moreover, we can get some details about the restaurants including price, utilities(ex: Wi-Fi), etc.. 

\subsection{Others}
More information about the area (by zip code) including average house price, criminal rate, etc..

\section{Overall Research Questions}

\begin{enumerate}
  \item Is there any relationship(positive or negative) between the restaurant scores by Health Department and consumers? Also, do they have  the same distribution?
  \item How does location influent on restaurants?
\end{enumerate}

\section{Approach}
We are going to use Chi-Squared Test and Linear Regression Model to exam the relation between two kinds of scores. For the model of restaurant location, we will try some General Lineal Models like Logistic Model.

\section{Anticipated results}



\section{Reference}




\end{document}