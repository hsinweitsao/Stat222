\documentclass{article}
\usepackage{amsmath}
\usepackage[margin=2cm]{geometry}
\usepackage{hyperref}
\usepackage{indentfirst}
\usepackage{setspace}
\doublespacing
\title{Stat222 Revised Research Proposal}
\author{Hui-Fang Ko, Hsin-Wei Tsao}
\date{Feb 19, 2014}
\begin{document}
\maketitle

\section{Introduction}
As consumers, we really want to have good meal experiences which are highly correlated to the restaurants. So we try to make a more solid model to evaluate restaurants not only based on the meal and service they provide, but also taking hygiene issues into account.



\section{Resources (Dataset)}
\subsection{Restaurant Scores in San Francisco}
This is a dataset provided by the San Francisco of Department of Public Health at \url{https://data.sfgov.org/Public-Health/Restaurant-Scores/stya-26eb}. The Health Department has developed an inspection report and scoring system. After conducting an inspection of the facility, the Health Inspector calculates a score based on the violations observed. So this dataset contains information of 6073 restaurants in San Francisco area and their score in the inspection conducted by the department.

\subsection{Yelp API}
With Yelp API, we can get the consumers' reviews of those restaurants listed in the dataset above. Moreover, we can get some details about the restaurants including price, utilities(ex: Wi-Fi), etc.. 

\subsection{Others}
More information about the area (by zip code) including average housing price, criminal rate, etc..

\section{Overall Research Questions}

\begin{enumerate}
  \item Is there any relationship(positive or negative) between the restaurant scores by Health Department and consumers' preferences on Yelp? Also, do they have  the same distribution?
  \item How does location influence on restaurants' score?
\end{enumerate}

\section{Approach}
We are going to use Chi-Squared Test and Linear Regression Model to exam the relation between two kinds of scores. For the model of restaurant location, we will try some General Lineal Models like Logistic Model.

\section{Anticipated Results}
\begin{enumerate}
  \item Those two scores might have positive relationship, but might not have same distribution.
  \item We believe that locations have good influence on restaurants. So maybe we could build a food map recommending nice restaurants with both high ratings on Yelp and nice, clean environment. 
\end{enumerate}


\section{Anticipated Figures \& Tables}
\begin{enumerate}
  \item The scope of relationship of 2 different scores.
  \item A table shows the weights that we used to evaluate a restaurants and the scores we give.
   \item The map of recommended area which has higher chance to get better restaurants. (By our scores.)
  \item A comparison with our scores and customers'. 
\end{enumerate}



\section{Reference}
\begin{enumerate}
  \item10 things to Know About Choosing a Restaurant Location \url{http://restaurants.about.com/od/location/a/10-Things-To-Know-About-Choosing-A-Restaurant-Location.htm}
  \item Consumers Feast on Restaurant Ratings \url{http://www.gsb.stanford.edu/news/research/stratman_consumerinfo.shtml}
  \item 8 Marketing Technologies that Affect Customer Restaurant Choices \url{http://www.foodservicewarehouse.com/education/restaurant-marketing/8-marketing-technologies-that-affect-customer-restaurant-choices/c28057.aspx}
\end{enumerate}




\end{document}